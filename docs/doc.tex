\documentclass[11pt, a4paper]{report}

% --- Language and Encoding ---
\usepackage[utf8]{inputenc}
\usepackage[T1]{fontenc}
\usepackage[english]{babel}

% --- Fonts (Sans-serif default) ---
\usepackage{microtype} % Improves text spacing

% --- Page Layout & Margins ---
\usepackage[a4paper, top=3.0cm, bottom=3.0cm, left=2.5cm, right=2.5cm, headheight=15pt]{geometry}

% --- Headers and Footers ---
\usepackage{fancyhdr}
\pagestyle{fancy}
\fancyhf{} % Clear all defaults

% Header
\rhead{\thepage}

% Footer
\lfoot{\footnotesize \textit{Converged Engineering Homelab}} % Project Title
% Line thickness
\renewcommand{\headrulewidth}{0.4pt}
\renewcommand{\footrulewidth}{0.4pt}

% Style for Chapter pages (plain style usually suppresses headers, we override it)
\fancypagestyle{plain}{
\fancyhf{}
\rhead{\thepage}
\lfoot{\footnotesize \textit{Converged Engineering Homelab}}
\renewcommand{\headrulewidth}{0.4pt}
\renewcommand{\footrulewidth}{0.4pt}
}

% --- Chapter/Section Titles Styling ---
\usepackage{titlesec}
\usepackage{color}

% Chapter Format
\titleformat{\chapter}[display]
{\normalfont\huge\bfseries} % Format
{\chaptertitlename\ \thechapter} % Label (e.g., Capitolo 1)
{20pt} % Separation
{\Huge} % Title size

% Section Format
\titleformat{\section}
{\normalfont\Large\bfseries}
{\thesection}
{1em}
{}

% --- Graphics & Tables ---
\usepackage{graphicx}
\usepackage{booktabs}
\usepackage{array}

% --- Code Snippets ---
\usepackage{minted}
\usepackage{xcolor}
\usemintedstyle{xcode}
\setminted{
bgcolor=lightgray!20,
framesep=2mm,
xleftmargin=10pt,
fontsize=\small,
}

% --- Hyperlinks ---
\usepackage[hidelinks]{hyperref}

\usepackage{fontspec}
\setmainfont{IBM Plex Sans}
\setmonofont{IBM Plex Mono}
\newfontfamily\titlefont{IBM Plex Serif}

\usepackage{listings}
\usepackage{xcolor}
\usepackage{geometry}
\geometry{margin=2.5cm}

\usepackage[most]{tcolorbox}

\usepackage[most]{tcolorbox}

\newtcbox{\statusbubble}[1][gray]{
  on line, 
  arc=12pt, % Adjust for "pill" roundness
  before upper=\strut, 
  colback=#1!20, % Light version of the color for background
  colframe=#1!20, % Darker version for the border
  colupper=black!80, % This sets the text color correctly
  boxrule=0pt, 
  left=8pt, right=8pt, top=4pt, bottom=4pt, % Padding
  boxsep=0pt,
  %fontupper=\sffamily\scriptsize\bfseries % Style of the text inside
}

% Define the specific \draft command
\newcommand{\draft}{\statusbubble[gray]{DRAFT}}
\newcommand{\inprogress}{\statusbubble[orange]{IN PROGRESS}}
\newcommand{\review}{\statusbubble[blue]{REVIEW}}
\newcommand{\done}{\statusbubble[green]{DONE}}

\begin{document}

% --- Custom Title Page ---
\begin{titlepage}
\newgeometry{top=2cm, bottom=2cm, left=2.5cm, right=2.5cm}

\vspace*{2.5cm}

% Title Region
\begin{flushleft}
\titlefont 
    {\Huge Converged Engineering Homelab} \\[1cm]
\end{flushleft}
\vspace{1.2cm}

\hrule height 1pt % Thicker main divider line
\vspace{8pt} % Ensures alignment at the top

% Helper command for consistent formatting
% Usage: \field{Label}{Content}
\newcommand{\field}[2]{%
    \noindent{\footnotesize #1}\\[8pt] % Label (Small)
    {\Large \textbf{#2}}\par           % Content (Large & Bold)
}

% --- ROW 1: People ---
\noindent
\begin{minipage}[t]{0.48\textwidth} 
    \vspace{0pt} % Ensures alignment at the top
    \field{Assignee(s)}{Mazza Luca}
\end{minipage}%
\hfill
\begin{minipage}[t]{0.48\textwidth}
    \vspace{0pt}
    \field{Version}{v0.0.1}
    
    \vspace{35pt}\hrule height 1pt\vspace{8pt} % Line separator
    
    \field{Committer}{-}
    
    \vspace{35pt}\hrule height 1pt\vspace{8pt} % Line separator
    
    \field{Status}{\draft}
\end{minipage}

\vspace{40pt}

% --- ROW 2: Course Info ---
\hrule height 1pt % Thicker main divider line
\vspace{8pt}

\noindent
\begin{minipage}[t]{0.525\textwidth}
    \vspace{0pt}
    \field{Type}{Systems Management \& DevOps}
\end{minipage}%
\hfill
\begin{minipage}[t]{0.6\textwidth}
    \vspace{0pt}
    \field{Code}{P17-001}
\end{minipage}

\vspace{35pt}
\hrule height 1pt % Thicker main divider line
\vspace{8pt}

% --- ROW 3: Date ---
\noindent
\field{Year}{2025 / 2026}
\vspace{5cm} % Space at the very bottom
\hrule height 1pt width 0.5\textwidth % Thicker main divider line
\vspace{8pt}
\noindent
\begin{minipage}[t]{\textwidth}
    \vspace{0pt}
    \field{Date}{30th May 2026}
\end{minipage}
\end{titlepage}
\restoregeometry

% --- Front Matter ---
\tableofcontents
\newpage

% --- List of Figures/Tables (Optional) ---
\listoffigures
\listoftables
\newpage

% --- Abstract (As Chapter 1 in source) ---
\chapter{Abstract}

This project consolidates storage, development, and high-end gaming into a single 4U rackmount server.
It uses Proxmox VE and PCIe Passthrough to dynamically assign hardware to virtual machines. 
The network is managed by a low-power Gateway (Mac Mini M1) and controlled via a high-end client
(Mac Mini M4).

The grounds for this project are the need for a powerful, flexible, and efficient home lab that can handle various
workloads without the need for multiple devices and the desire to setup and tear down environments on the fly, without 
the need to reboot or reconfigure hardware.

With this kind of infrastructure, I can easily switch between different operating systems and software configurations,
making it ideal for development, testing, and gaming.
Also switching between different development enviromnents through the use of containers and virtual machines, without 
the need to reboot or reconfigure hardware, is a key feature of this project.

\vspace{1cm}
\noindent\rule{\textwidth}{0.5pt}
\vspace{1cm}

% --- Content ---
\chapter{Introduction}

\section{Bill of Materials}

Here is a detailed bill of materials for the project; obviously the prices may vary based on retailer, availability and 
location, but this should give a pretty accurate estimate for the costs of the project's hardware components.

\textit{Note: the prices are shown in Swiss Francs (CHF) and components marked with (U) are used but in good condition.}

Some of the components, such as the switch and the Mac Mini M1 are recycled from my personal hardware collection, 
and the price listed is an estimate of what I paid for them when I bought them, but they are still in good condition and
fully functional, so they are a crucial part of the project.

\begin{table}[h]
\centering
\begin{tabular}{@{}lp{7cm}rp{4cm}l@{}}
\toprule
\textbf{Category} & \textbf{Description} & \textbf{Price} & \textbf{Notes} \\ \midrule
Chassis      & Lanberg SC01-5204-12B   & 129.— & 4U Rackmount Case\\
CPU          & Ryzen 7 7700            & 220.— & 65W TDP and iGPU\\
GPU          & nVidia RTX 5070         & 575.— & PNY\\
Motherboard  & AsRock B850 LiveMixer   & 164.— & AM5, ATX\\
RAM          & Crucial Pro             & 629.— & 64GB 5600MHz DDR5\\
Storage      & WD Black SN7100         & 151.— & 1TB M.2 2280\\
Storage      & Seagate IronWolf        & 276.— & 2x4TB 3.5" HDD\\
Fans         & Noctua NF-A12X25 PWM    &  31.— & 120mm\\
Rack Cabinet & Digitus DN-48000        &  88.— & 8RU 19"\\
Patch Panel  & Digitus DN93651         &  50.— & 24-Port Cat6\\
Rack Shelf   & Digitus DN-19 Tray      &  15.— & 19" Rack Shelf\\
Server       & Mac Mini M1             & 700.— & M1, 16GB RAM\\
Switch       & Aruba 2530-48G-PoE+ (U) & 400.— & 48-Port PoE+ Switch\\
Cables       & Various                 &  50.— & Ethernet, Power, etc.\\
Misc         & Miscellaneous           &  50.— & HDMI Dummy, Screws\\
\midrule
\textbf{Total} &  & \textbf{3528.—} & \\
\bottomrule
\end{tabular}
    \caption{Rack Components}
\end{table}

These are the materials used to build the core rack of the project, any other material (i.e. peripherals, consoles, 
etc...) are not crucial to the project and are not listed here. Any peripheral with a network interface can be used with
the rack, as long as it supports the necessary protocols and standards.

\section{Requisiti}
\begin{table}[h]
\centering
\begin{tabular}{@{}lp{8cm}l@{}}
\toprule
\textbf{ID} & \textbf{Descrizione} & \textbf{Note} \\ \midrule
R-01 & X & \\
R-02 & X & \\
R-04 & X & \\ 
\bottomrule
\end{tabular}
\caption{Requisiti del progetto}
\end{table}

\chapter{Design e implementazione}
\section{API}
Di seguito è riportato un esempio di codice per l'inizializzazione.

\begin{minted}{cpp}
// Esempio di inizializzazione
float x = 0.5f;
std::vector<float> input = {x, x, x}; // Input per il modello
// Altri parametri...
\end{minted}

\chapter{Conclusioni}

\end{document}
